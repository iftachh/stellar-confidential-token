\providecommand{\remove}[1]{}
\newcommand{\Draft}[1]{\ifdefined\IsDraft\texttt{ #1} \fi}
\newcommand\numberthis{\addtocounter{equation}{1}\tag{\theequation}}


\ifdefined\IsDraft
\newcommand{\authnote}[2]{{\bf [{\color{red} #1's Note:} {\color{blue} #2}]}}
\else
\newcommand{\authnote}[2]{}
\fi

\newcommand{\strongAuthNote}[1]{{\bf [{\color{purple} #1}]}}





\newcommand{\Ensuremath}[1]{\ensuremath{#1}\xspace}
\newcommand{\MathAlgX}[1]{\Ensuremath{\MathAlg{#1}}}
\newcommand{\myOptName}[1]{\Ensuremath{\operatorname{#1}}}
%Type macros%%%%%%%%%%%%%%%%%%%%%%%

%%% mathcal letters
\def\mydefc#1{\expandafter\def\csname c#1\endcsname{\mathcal{#1}}}
\def\mydefallc#1{\ifx#1\mydefallc\else\mydefc#1\expandafter\mydefallc\fi}
% \c* gives you \mathcal{*} where * can be any of the following letters
\mydefallc ABCDEFGHIJKLMNOPQRSTUVWXYZ\mydefallc


%%% overline letters
\def\mydefo#1{\expandafter\def\csname o#1\endcsname{\overline{#1}}}
\def\mydefallo#1{\ifx#1\mydefallo\else\mydefo#1\expandafter\mydefallo\fi}
\mydefallo aewABCDEFGHIJKLMNOPQRSTUVWXYZx\mydefallo

\newcommand{\orr}{\overline{r}}

%%% mathbf letters
\def\mydefbf#1{\expandafter\def\csname bf#1\endcsname{\mathbf{#1}}}
\def\mydefallbf#1{\ifx#1\mydefallbf\else\mydefbf#1\expandafter\mydefallbf\fi}
\mydefallbf abcdefghijklmnopqrstuvwxyzABCDEFGHIJKLMNOPQRSTUVWXYZ\mydefallbf


%%% tilde letters
\def\mydeftilde#1{\expandafter\def\csname t#1\endcsname{\widetilde{#1}}}
\def\mydefalltilde#1{\ifx#1\mydefalltilde\else\mydeftilde#1\expandafter\mydefalltilde\fi}
\mydefalltilde abijkqrftABCHGDSXYKMRWTQt\mydefalltilde


%%% hat letters
\def\mydefhat#1{\expandafter\def\csname h#1\endcsname{\widehat{#1}}}
\def\mydefallhat#1{\ifx#1\mydefallhat\else\mydefhat#1\expandafter\mydefallhat\fi}
\mydefallhat ABCDEFGHIJKMNOPSRTQUVWXYZabhorgq\mydefallhat


%%% Alg letters
\def\mydefAlg#1{\expandafter\def\csname #1c\endcsname{\MathAlgX{#1}}}
\def\mydefallAlg#1{\ifx#1\mydefallAlg\else\mydefAlg#1\expandafter\mydefallAlg\fi}
\mydefallAlg ABCDEFGHIJKMNOPQRSTUVWXYZ\mydefallAlg

%%% hAlg letters
\def\mydefHAlg#1{\expandafter\def\csname h#1c\endcsname{\MathAlgX{\widehat{#1}}}}
\def\mydefallHAlg#1{\ifx#1\mydefallHAlg\else\mydefHAlg#1\expandafter\mydefallHAlg\fi}
\mydefallHAlg ABCDEFGHIJKMNOPQRSTUVWXYZ\mydefallHAlg

%%% oAlg letters
\def\mydefOAlg#1{\expandafter\def\csname o#1c\endcsname{\MathAlgX{\overline{#1}}}}
\def\mydefallOAlg#1{\ifx#1\mydefallOAlg\else\mydefOAlg#1\expandafter\mydefallOAlg\fi}
\mydefallOAlg ABCDEFGHIJKMNOPQRSTUVWXYZ\mydefallOAlg


%%% tAlg letters
\def\mydefTAlg#1{\expandafter\def\csname t#1c\endcsname{\MathAlgX{\widetilde{#1}}}}
\def\mydefallTAlg#1{\ifx#1\mydefallTAlg\else\mydefTAlg#1\expandafter\mydefallTAlg\fi}
\mydefallTAlg ABCDEFGHIJKMNOPQRSTUVWXYZ\mydefallTAlg


\newcommand{\zns}{\Z_n^\ast}

%Cref%%%%%%%%%%%%%%%%

\newcommand{\sdotfill}{\textcolor[rgb]{0.8,0.8,0.8}{\dotfill}} %change to \cdotfill later

%\newenvironment{protocol}{\begin{proto}}{\vspace{-\topsep}~\newline\sdotfill\end{proto}}

\newenvironment{protocol}
{\begin{mymdframed}\begin{proto}}
		{\end{proto}\end{mymdframed}}


%\newenvironment{protocol}{\begin{mybox} \vspace{-.1in}\begin{proto}}{ \vspace{-.1in} \end{proto}\end{mybox}}

%\newenvironment{functionality}{\begin{funct}}{\vspace{-\topsep}\sdotfill\end{funct}}

\newenvironment{functionality}
{\begin{mymdframed}\begin{funct}}
		{\end{funct}\end{mymdframed}}
	
%\newenvironment{functionality}{\begin{mybox} \vspace{-.1in}\begin{funct}}{ \vspace{-.1in} \end{funct}\end{mybox}}

%\newenvironment{algorithm}
%{\begin{algo}}
%{\vspace{-\topsep}~\sdotfill\end{algo}}



\definecolor{mylightgrey}{RGB}{245, 245, 245}

\newenvironment{mymdframed}
{
	\begin{mdframed}[%
	%outerlinewidth = 2,%
	roundcorner = 10pt,%
	innertopmargin = 1,%
	%leftmargin = 40,%
	%leftmargin = 40,%
	%rightmargin = 40,%
	backgroundcolor = mylightgrey,%
	%outerlinecolor = blue!70! black,%
	%	innertopmargin = \topskip,%
	%	splittopskip = \topskip,%
	%ntheorem = true
	] 
}
{
\end{mdframed}
}


\newenvironment{game}
{\begin{mymdframed}\begin{gamo}}
		{\end{gamo}\end{mymdframed}}


\newenvironment{algorithm}
{\begin{mymdframed}\begin{algo}}
{\end{algo}\end{mymdframed}}


%\newenvironment{algorithm}{\begin{mybox} \vspace{-.1in}\begin{algo}}{ \vspace{-.1in} \end{algo}\end{mybox}}
\newenvironment{InlinedAlgorithm}{\begin{algo}}{\end{algo}}


\newenvironment{experiment}{\begin{expr}}{\end{expr}}

\newenvironment{mybox}{\begin{center}\begin{tabular}{|p{0.97\linewidth}|c|}   \hline} {  \\ \hline \end{tabular} \end{center}}			



%\newenvironment{algo}{\begin{center}\setlength{\fboxsep}{12pt}
		%\begin{boxedminipage}{\linewidth}}
		%{\end{boxedminipage} \end{center}}

%%% General %%%%%%%%%%%%%%%%%%%%%%%%%%%%%%%%%%%%%%%%%%%%%%%%%%%%%%

%fix spacing issues with \left \right
\let\originalleft\left
\let\originalright\right
\renewcommand{\left}{\mathopen{}\mathclose\bgroup\originalleft}
\renewcommand{\right}{\aftergroup\egroup\originalright}


%\newcommand{\etal}{{et~al.\xspace}
	\newcommand{\aka} {also known as,\xspace}
	\newcommand{\resp}{resp.,\xspace}
	\newcommand{\ie}  {i.e.,\xspace}
	\newcommand{\eg}  {e.g.,\xspace}
	\newcommand{\whp}  {with high probability\xspace}
	\newcommand{\wrt} {with respect to\xspace}
	\newcommand{\wlg} {without loss of generality\xspace}
	\newcommand{\Wlg} {Without loss of generality\xspace}
	\newcommand{\cf}{cf.,\xspace}
	\newcommand{\vecc}[1]{\left\lVert #1 \right\rVert}
	\newcommand{\abs}[1]{\left\lvert #1 \right\rvert}
	\newcommand{\ceil}[1]{\left\lceil #1 \right\rceil}
	\newcommand{\ip}[1]{\iprod{#1}}
	\newcommand{\iprod}[1]{\langle #1 \rangle}
	\newcommand{\seq}[1]{\langle #1 \rangle}
	\newcommand{\sset}[1]{\ens{#1}}
	\newcommand{\set}[1]{ \left\{#1 \right\}}
	%\newcommand{\expp}[1]{\exp\left( #1\right)}
	\newcommand{\expp}[1]{e^{#1}}
	\newcommand{\paren}[1]{ (#1 )}
	\newcommand{\LRparen}[1]{ \left(#1 \right)}
	\newcommand{\divs}{\;|\;}
	\newcommand{\ynote}[1]{\authnote{Yuval}{#1}}
	\newcommand{\inote}[1]{\authnote{Iftach}{#1}} 
	\newcommand{\nnote}[1]{\authnote{Nikos}{#1}}
	
	\newcommand{\Brack}[1]{{\left[#1\right]}}
	\newcommand{\inner}[2]{\left[#1,#2\right]}
	\newcommand{\floor}[1]{\left \lfloor#1 \right \rfloor}
	%\newcommand{\vect}[1]{{ \mathbf #1}}
	%\newcommand{\mat}[1]{{ \mathbf #1}}
	\newcommand{\lat}{\mathcal{L}}
	\newcommand{\latP}[1]{\lat\paren{#1}}
	\newcommand{\norm}[1]{\left\lVert#1\right\rVert}
	
	
	\newcommand{\cond}{\;|\;}
	\newcommand{\half}{\tfrac{1}{2}}
	\newcommand{\third}{\tfrac{1}{3}}
	\newcommand{\quarter}{\tfrac{1}{4}}
	%\newcommand{\eqdef}{\mathbin{{\rm def}{=}}}
	\newcommand{\eqdef}{:=}
	
	
	\newcommand{\N}{{\mathbb{N}}}
	\newcommand{\Z}{{\mathbb Z}}
	\newcommand{\F}{{\cal F}}
	\newcommand{\io}{\hyclass{i.o. \negmedspace-}}
	\newcommand{\naturals}{{\mathbb{N}}}
	\newcommand{\zo}{\set{0,1}}
	\newcommand{\oo}{\mo}
	
	\newcommand{\mo}{\set{\!\shortminus 1,1\!}}
	\newcommand{\mon}{\mo^n}
	
	
	\newcommand{\zn}{{\zo^n}}
	\newcommand{\zs}{{\zo^\ast}}
	\newcommand{\zl}{{\zo^\ell}}
	
	\newcommand{\suchthat}{{\;\; : \;\;}}
	\newcommand{\condition}{\;\ifnum\currentgrouptype=16 \middle\fi|\;}
	\newcommand{\deffont}{\em}
	\let\gets\undefined
	\newcommand{\getsr}{\mathbin{\stackrel{\mbox{\tiny R}}{\leftarrow}}}
	
	
	
	\newcommand{\IsIn}{\mathbin{\stackrel{ ?}{\in}}}
	
	
	%\newcommand{\getsr}{\gets}
	\newcommand{\xor}{\oplus}
	\newcommand{\al}{\alpha}
	\newcommand{\be}{\beta}
	\newcommand{\eps}{\varepsilon}
	\newcommand{\ci} {\stackrel{\rm{c}}{\equiv}}
	\newcommand{\from}{\leftarrow}
	\newcommand{\la}{\leftarrow}
	
	\newcommand{\negfunc}{\e}
	
	
	%%%%%%%%%%%%%%%%%%% functions definition %%%%%%%%%%%%%%%%%%%%%%%%%%%%%%
	
	\newcommand{\Enc}{\MathAlgX{Enc}}
\newcommand{\Dec}{\MathAlgX{Dec}}
%\newcommand{\Gen}{\MathAlgX{Gen}}
\newcommand{\KeyGen}{\MathAlgX{KeyGen}}

%\newcommand{\GenE}{\MathAlgX{ExtdGen}}
	
	
	% need to replace \operatorname with \newfunc
	\newcommand{\diver}{D}
	\newcommand{\re}[2]{\diver\paren{#1 || #2}}
	
	\newcommand{\loglog}{\operatorname{loglog}}
	\newcommand{\logloglog}{\operatorname{logloglog}}
	

	\newcommand{\Hall}{\operatorname{H}}
	\newcommand{\Hmin}{\operatorname{H_{\infty}}}
	\newcommand{\HRen}{\operatorname{H_2}}
	\newcommand{\Ext}{\MathAlgX{Ext}}
	\newcommand{\Con}{\operatorname{Con}}
	\newcommand{\Samp}{\operatorname{Smp}}

	\newcommand{\Time}{\operatorname{time}}
	\newcommand{\negl}{\operatorname{neg}}
	\newcommand{\Supp}{\operatorname{Supp}}
	\newcommand{\maj}{\operatorname*{maj}}
	\newcommand{\argmax}{\operatorname*{argmax}}
	\newcommand{\argmin}{\operatorname*{argmin}}
	
	%\newcommand{\accept}{\mathtt{accept}}
	%\newcommand{\reject}{\mathtt{reject}}
	\newcommand{\accept}{\ensuremath{\mathtt{Accept}}}
	\newcommand{\reject}{\ensuremath{\mathtt{Reject}}}
	
	%\newcommand{\fail}{\mathtt{fail}}
	\newcommand{\fail}{\mathsf{fail}}
	
	\newcommand{\halt}{\mathtt{halt}}
	
	\newcommand{\MathFam}[1]{\mathcal{#1}}
	\newcommand{\ComFam}{\MathFam{COM}}
	\newcommand{\HFam}{\MathFam{H}}
	\newcommand{\FFam}{\MathFam{F}}
	\newcommand{\GFam}{\MathFam{\GT}}
	\newcommand{\Dom}{\MathFam{D}}
	\newcommand{\Rng}{\MathFam{R}}
	\newcommand{\Code}{\MathFam{C}}
	\newcommand{\ndiv}{\,\cancel{\:|\,}\: }
	\newcommand{\MathAlg}[1]{\mathsf{#1}}
	\newcommand{\Com}{\MathAlg{Com}}
	\newcommand{\sign}{\MathAlg{sign}}
	
	\newcommand{\zfrac}[2]{#1/#2}
	
	
	
	\newcommand{\tth}[1]{\Ensuremath{#1^{\rm th}}}
	\newcommand{\tthsp}[1]{\Ensuremath{#1^{\rm th}}}
	\newcommand{\ith}{\tth{i}}
	\newcommand{\mth}{\tth{m}}
	\newcommand{\jth}{\tth{j}}
		\newcommand{\dth}{\tth{d}}
		
	\newcommand{\brackets}[1]{\bigl(#1\bigr)}
	\newcommand{\bracketss}[1]{\left(#1\right)}
	
	
	%%% Computational Problems %%%%%%%%%%%%%%%%%%%%%%%%%%%%%%%%%%%%%%%
	
	\def\textprob#1{\textmd{\textsc{#1}}}
	\newcommand{\mathprob}[1]{\mbox{\textmd{\textsc{#1}}}}
	%\newcommand{\SAT}{\mathprob{SAT}}
	\newcommand{\HamPath}{\textprob{Hamiltonian Path}}
	\newcommand{\StatDiff}{\textprob{Statistical Difference}}
	\newcommand{\EntDiff}{\textprob{Entropy Difference}}
	\newcommand{\EntApprox}{\textprob{Entropy Approximation}}
	\newcommand{\CondEntApprox}{\textprob{Conditional Entropy Approximation}}
	\newcommand{\QuadRes}{\textprob{Quadratic Residuosity}}
	\newcommand{\CktApprox}{\mathprob{Circuit-Approx}}
	\newcommand{\DNFRelApprox}{\mathprob{DNF-RelApprox}}
	\newcommand{\GraphNoniso}{\textprob{Graph Nonisomorphism}}
	\newcommand{\GNI}{\mathprob{GNI}}
	\newcommand{\GraphIso}{\textprob{Graph Isomorphism}}
	%\newcommand{\GI}{\mathprob{Graph Isomorphism}}
	\newcommand{\MinCut}{\textprob{Min-Cut}}
	\newcommand{\MaxCut}{\textprob{Max-Cut}}
	
	\newcommand{\yes}{YES}%{{\sc yes}}
	\newcommand{\no}{NO}%{{\sc no}}
	
	
	%%% Complexity Classes %%%%%%%%%%%%%%%%%%%%%%%%%%%%%%%%%%%%%%%%%%
	
	%Cref isuues
	%make all reference start with uppercase
	\renewcommand{\cref}{\Cref}
	\usepackage{aliascnt}
	
	
	\newcommand{\itemref}[1]{Item~\ref{#1}}
	\newcommand{\roundref}[1]{Round~\ref{#1}}
	
	
	%\ifCR\else
	%\theoremstyle{nicetheorem}
	%\fi
	
	
	
	\newtheorem{theorem}{Theorem}[section]
	
	

\remove{	
	\newaliascnt{mythm}{theorem}
	\newtheorem{mythm}[mythm]{Theorem}
	\aliascntresetthe{mythm}
	
	\renewenvironment{theorem}
	{\begin{mymdframed}\begin{mythm}}
			{\end{mythm}\end{mymdframed}}
}	
	
	
	\newaliascnt{lemma}{theorem}
	\newtheorem{lemma}[lemma]{Lemma}
	\aliascntresetthe{lemma}
	\crefname{lemma}{Lemma}{Lemmas}
	
	
	
	%\providecommand*{\lemmaautorefname}{Lemma}
	\newaliascnt{claim}{theorem}
	\newtheorem{claim}[claim]{Claim}
	\aliascntresetthe{claim}
	\crefname{claim}{Claim}{Claims}
	
	\newaliascnt{corollary}{theorem}
	\newtheorem{corollary}[corollary]{Corollary}
	\aliascntresetthe{corollary}
	\crefname{corollary}{Corollary}{Corollaries}
	
	
	\newaliascnt{construction}{theorem}
	\newtheorem{construction}[construction]{Construction}
	\aliascntresetthe{construction}
	\crefname{construction}{Construction}{Constructions}
	
	\newaliascnt{fact}{theorem}
	\newtheorem{fact}[fact]{Fact}
	\aliascntresetthe{fact}
	\crefname{fact}{Fact}{Facts}
	
	\newaliascnt{proposition}{theorem}
	\newtheorem{proposition}[proposition]{Proposition}
	\aliascntresetthe{proposition}
	\crefname{proposition}{Proposition}{Propositions}
	
	\newaliascnt{conjecture}{theorem}
	\newtheorem{conjecture}[conjecture]{Conjecture}
	\aliascntresetthe{conjecture}
	\crefname{conjecture}{Conjecture}{Conjectures}
	
	
	\newaliascnt{definition}{theorem}
	\newtheorem{definition}[definition]{Definition}
	\aliascntresetthe{definition}
	\crefname{definition}{Definition}{Definitions}
	
	\newaliascnt{notation}{theorem}
	\newtheorem{notation}[notation]{Notation}
	\aliascntresetthe{notation}
	\crefname{notation}{Notation}{Notation}
	
	\newaliascnt{assertion}{theorem}
	\newtheorem{assertion}[assertion]{Assertion}
	\aliascntresetthe{assertion}
	\crefname{assertion}{Assertion}{Assertion}
	
	\newaliascnt{assumption}{theorem}
	\newtheorem{assumption}[assumption]{Assumption}
	\aliascntresetthe{assumption}
	\crefname{assumption}{Assumption}{Assumption}
	
%	\theoremstyle{remark}
	\newaliascnt{remark}{theorem}
	\newtheorem{remark}[remark]{Remark}
	\aliascntresetthe{remark}
	\crefname{remark}{Remark}{Remarks}
	
	%\theoremstyle{plain}
	\newaliascnt{question}{theorem}
	\newtheorem{question}[question]{Question}
	\aliascntresetthe{question}
	\crefname{question}{Question}{Questions}
	
	%\theoremstyle{definition}
	\newaliascnt{example}{theorem}
	\ifdefined\excludeexample
	\else
	\newtheorem{example}[example]{Example}
	\fi
	
	
	\aliascntresetthe{example}
	\crefname{exmaple}{Example}{Examples}
	%\theoremstyle{plain}
	
	\crefname{equation}{Equation}{Equations}
	
	
	%%%%%%%%%%%%%%%%%%%%%%%%%%%%%%%%%%%%%%%%%%%%%%%%%%%%%%%%%%%%%%%%%%%%%%%%%
	%\theoremstyle{protocol}
	%what is proto
	\newaliascnt{proto}{theorem}
	
	%the name to appear for the environ
	\theoremstyle{definition}
	\newtheorem{proto}[proto]{Protocol}
	
	%the name to appear in the reference
	\aliascntresetthe{proto}
	\crefname{proto}{protocol}{protocols}
	%%%%%%%%%%%%%%%%%%%%%%%%%%%%%%%%%%%%%%%%%%%%%%%%%%%%%%%%
	
	
	
	\newaliascnt{algo}{theorem}
	\theoremstyle{plain}
	\newtheorem{algo}[algo]{Algorithm}
	\aliascntresetthe{algo}
	
	
		\newaliascnt{gamo}{theorem}
	\theoremstyle{plain}
	\newtheorem{gamo}[gamo]{Game}
	\aliascntresetthe{gamo}
	
	%\crefname{algorithm}{algorithm}{algorithms}
	%\crefname{algo}{algorithm}{algorithms}
	
	\newaliascnt{funct}{theorem}
	\theoremstyle{definition}
	\newtheorem{funct}[funct]{Functionality}
	\aliascntresetthe{funct}
	\crefname{funct}{functionality}{Functionality}
	
	
	\newaliascnt{expr}{theorem}
	\theoremstyle{plain}
	\newtheorem{expr}[expr]{Experiment}
	\aliascntresetthe{expr}
	\crefname{experiment}{experiment}{experiments}
	
	
	%
	\newcommand{\stepref}[1]{Step~\ref{#1}}
	\newcommand{\progref}[1]{Program~\ref{#1}}
	
	
	
	
	%%% Proof environments %%%%%%%%%%%%%%%%%%%%%%%%%%%%%%%%%%%%%%%%%%%
	
	\def\FullBox{$\Box$}
	\def\qed{\ifmmode\qquad\FullBox\else{\unskip\nobreak\hfil
			\penalty50\hskip1em\null\nobreak\hfil\FullBox
			\parfillskip=0pt\finalhyphendemerits=0\endgraf}\fi}
	
	\def\qedsketch{\ifmmode\Box\else{\unskip\nobreak\hfil
			\penalty50\hskip1em\null\nobreak\hfil$\Box$
			\parfillskip=0pt\finalhyphendemerits=0\endgraf}\fi}
	
	\newenvironment{proofidea}{\begin{trivlist} \item {\it
				Proof Idea.}} {\end{trivlist}}
	
	%\newenvironment{proofof}[1]{\begin{proof}[Proof of~#1]}{\end{proof}}
	
	\newenvironment{proofsketch}{\begin{trivlist} \item {\it
				Proof sketch.}} {\qed\end{trivlist}}
	
	\newenvironment{proofskof}[1]{\begin{trivlist} \item {\it Proof
				Sketch of~#1.}} {\end{trivlist}}
	
	

	
	%\newcommand{\simgeq}{\; \raisebox{-0.4ex}{\tiny$\stackrel
	%		{{\textstyle>}}{\sim}$}\;}
%	\newcommand{\simleq}{\; \raisebox{-0.4ex}{\tiny$\stackrel
%			{{\textstyle<}}{\sim}$}\;}
	
	
	%\newcommand{\sex}[1]{\mathbf E [#1]}
	%\newcommand{\Ex}[1]{\Exp \left[#1\right]}
	%\newcommand{\Ex}[1]{{\mathbf E}\left[#1\right]}

	
		
 \DeclareMathOperator*{\Ex}{ E} 
		
	\newcommand{\eex}[2]{\Ex_{#1} \left[#2 \right]}
	\newcommand{\ex}[1]{\Ex [#1 ]}
	
	
%	\renewcommand{\Pr}{{\mathrm {Pr}}}
	
	\let\Pr\undefined
	
	\DeclareMathOperator*{\Pr}{Pr} %in preamble
	
	
	\newcommand{\pr}[1]{\Pr\left[#1\right]}
	\newcommand{\spr}[1]{\Pr[#1 ]}
	
	
	
	%\newcommand{\ppr}[2]{\underset{#1}{\Pr}\left[#2\right]}
	
	\newcommand{\ppr}[2]{\Pr_{#1}\left[#2\right]}
		
	
	
	
	
	
	
	\newcommand{\prob}[1]{\mathsf{\textsc{#1}}}
	

	
	
	
	
	\newcommand{\ens}[1]{\{#1\}}
	\newcommand{\size}[1]{\left|#1\right|}
	\newcommand{\ssize}[1]{|#1|}
	\newcommand{\bsize}[1]{\bigr|#1\bigr|}
	\newcommand{\enss}[1]{\{#1\}}
	\newcommand{\cindist}{\approx_c}
	\newcommand{\sindist}{\approx_s}

	\newcommand{\view}{{\operatorname{view}}}
	\newcommand{\View}{\operatorname{View}}
	\newcommand{\indist}{\approx}
	\newcommand{\dist}[1]{\mathbin{\stackrel{\rm {#1}}{\equiv}}}
	\newcommand{\Uni}{{\mathord{\mathcal{U}}}}
	
	
	%%% Zero Knowledge Characterization %%%%%%%%%%%%%%%%%%%%%%%%%%%%%%
	
	\newcommand{\OWF}{OWF}
	\newcommand{\OWFno}{OWF NO}
	\newcommand{\OWFyes}{OWF YES}
	
	\newcommand{\Sbb}{S}
	
	\newcommand{\yesinstance}{\mathrm{Y}}
	\newcommand{\noinstance}{\mathrm{N}}
	
	\newcommand{\SD}{\prob{SD}}
	\newcommand{\SDy}{\SD_\yesinstance}
	\newcommand{\SDn}{\SD_\noinstance}

	
	
	\newcommand{\1}{\mathds{1}}
	
	
		
	
	
	\newcommand{\KL}{\operatorname{KL}}

	\newcommand{\pptm}{{\sc pptm}\xspace}
	\newcommand{\ppt}{{\sc ppt}\xspace}
	
	
		
	
	
	
	\newcommand{\conc}{\circ}
	
	
	
	\newcommand{\Tableofcontents}{
		\thispagestyle{empty}
		\pagenumbering{gobble}
		\clearpage
		\tableofcontents
		\thispagestyle{empty}
		\clearpage
		\pagenumbering{arabic}
	}
	
	\newcommand{\vect}[1]{{ \bf #1}}
	\newcommand{\Mat}[1]{{ \bf #1}}
	\newcommand{\zot}{\set{0,1,2}}
	
	\newcommand{\Sim}{\MathAlgX{Sim}}
	
	%%%%%%%%%%%%%%%%
	
	
	
	

		
	
	\newcommand{\pk} {{\mathsf{pk}}}
	\newcommand{\sk} {{\mathsf{sk}}}
	
	
	
	
	\newcommand{\zer}{\mathbf{0}}
	\newcommand{\one}{\mathbf{1}}
	\newcommand{\ham}{\mathrm{dist}}
	\newcommand{\bT}{\boldsymbol{T}}
	\newcommand{\bU}{\boldsymbol{U}}
	\renewcommand{\u}{\boldsymbol{u}}
	\newcommand{\smthg}{\mathrm{smthg}}
	\newcommand{\f}{\boldsymbol{f}}
	\newcommand{\bado}{\cB_{1}}
	\newcommand{\badp}{\cB_{2}}
	\newcommand{\badt}{\cD}
	\newcommand{\badtt}{{\cF}}
	\newcommand{\badf}{\cB_{3}}
	\newcommand{\mat}[1]{\begin{pmatrix}#1\end{pmatrix}}
	
	\newcommand{\optional}{\textbf{optional}\xspace}
	
	
	\newcommand{\verify}[1]{Verify ~#1.}
	
	\newcommand{\asn}{\leftarrow}
	
	\newcommand{\Ham}{\operatorname{Ham}}
	
	%%%%%%%% NEW %%%%%%
	

	
	
		
	
	
	
	\newcommand{\ots}{\mathsf{OT}}
	\newcommand{\ot}[1]{\MathAlgX{\ots}_{#1}}
	
	


%% ---------------------- Session Macros
\newcommand{\sid}{\mathsf{sid}}

%% ---------------------- Party Macros






%%% mathcal letters
\def\mydefc#1{\expandafter\def\csname c#1\endcsname{\mathcal{#1}}}
\def\mydefallc#1{\ifx#1\mydefallc\else\mydefc#1\expandafter\mydefallc\fi}
% \c* gives you \mathcal{*} where * can be any of the following letters
\mydefallc ABCDEFGHIJKLMNOPQRSTUVWXYZ\mydefallc

%%% mathbf letters
\def\mydefbf#1{\expandafter\def\csname bf#1\endcsname{\mathbf{#1}}}
\def\mydefallbf#1{\ifx#1\mydefallbf\else\mydefbf#1\expandafter\mydefallbf\fi}
\mydefallbf abcdefghijklmnopqrstuvwxyzABCDEFGHIJKLMNOPQRSTUVWXYZ\mydefallbf

\newcommand{\cPKI}{\mathcal{PKI}}

%%% tilde letters
\def\mydeftilde#1{\expandafter\def\csname t#1\endcsname{\widetilde{#1}}}
\def\mydefalltilde#1{\ifx#1\mydefalltilde\else\mydeftilde#1\expandafter\mydefalltilde\fi}
\mydefalltilde abijkqrfABCHGDSXYKW\mydefalltilde

%%% bar letters
\def\mydefbar#1{\expandafter\def\csname br#1\endcsname{\bar{#1}}}
\def\mydefallbar#1{\ifx#1\mydefallbar\else\mydefbar#1\expandafter\mydefallbar\fi}
\mydefallbar GRe \mydefallbar



\mathchardef\mhyphen="2D

\newcommand{\onetwo}{\set{1,2}}
\newcommand{\bin}{\{0,1\}}
\newcommand{\bitset}{\bin}
\newcommand{\zm}{\zo^m}
\newcommand{\zkp}{\zo^\kappa}

\newcommand{\Q}{{\mathbb{Q}}}

\renewcommand{\G}{\gG}
\newcommand{\gG}{{\mathbb{G}}}
\newcommand{\tgG}{\widetilde{\gG}}
\newcommand{\hgG}{\widehat{\gG}}


\newcommand{\zp}{\mathbb{Z}_p}
\newcommand{\zps}{\zp^\ast}
\newcommand{\zq}{\mathbb{Z}_q}

\newcommand{\hbG}{\mathbb{\widehat{G}}}
\newcommand{\tbG}{\mathbb{\widetilde{G}}}

\newcommand{\angles}[1]{{\langle #1 \rangle}}
\newcommand{\send}{\rightarrow}
\newcommand{\sample}{\in_R}
\newcommand{\lagrange}{\mathsf{Lagrange}}
\newcommand{\Zn}{\Z_n}
\newcommand{\Zns}{\Z_n^\ast}

\newcommand{\QR}{\mathcal{QR}}
\newcommand{\ModuliGen}{\MathAlgX{ModuliGen}}
\newcommand{\StrongModuliGen}{\MathAlgX{StModuliGen}}

\newcommand{\false}{\myOptName{false}}
\newcommand{\true}{\myOptName{true}}
\newcommand{\Yes}{\myOptName{Yes}}
\newcommand{\No}{\myOptName{No}}
\newcommand{\Slack}{\myOptName{Slack}}


\newcommand{\bigbrack}{\vphantom{2^{2^2}}}
\newcommand{\medbrack}{\vphantom{\hat{1}^1}}


\newcommand{\hBeta}{\widehat{\beta}}
\newcommand{\hrho}{\widehat{\rho}}


\newcommand{\tE}{\widetilde{E}}
\newcommand{\tF}{\widetilde{F}}
\newcommand{\tdelta}{\widetilde{\delta}}
\newcommand{\tsigma}{\widetilde{\sigma}}
%\newcommand{\trho}{\widetilde{\rho}}

\newcommand{\tBeta}{\widetilde{\beta}}
\newcommand{\trho}{\widetilde{\rho}}
\newcommand{\tgamma}{\widetilde{\gamma}}


\newcommand{\Img}{\MathAlgX{Im}}


%% ----------------- Diagrams
\newcommand{\verylongleftarrow}[1]
{\setlength{\unitlength}{.01in}
    \begin{picture}(#1,1) \put(#1,0){\vector(-1,0){#1}} \end{picture}}
\newcommand{\verylongrightarrow}[1]
{\setlength{\unitlength}{.01in}
    \begin{picture}(#1,1) \put(0,0){\vector(1,0){#1}} \end{picture}}
\newcommand{\leftgoing}[2]{
    {\stackrel{{\displaystyle #2}} {\verylongleftarrow{#1}}}}
\newcommand{\rightgoing}[2]{
    {\stackrel{{\displaystyle #2}} {\verylongrightarrow{#1}}}}
\newcommand{\leftgoingc}[1]{\leftgoing{40}{#1} }
\newcommand{\vleftgoingc}[1]{\leftgoing{80}{#1} }
\newcommand{\rightgoingc}[1]{\rightgoing{40}{#1} }
\newcommand{\vrightgoingc}[1]{\rightgoing{80}{#1} }
\newcommand{\newss}[1]{{\scriptsize #1}}

%% ----------------- Simulator Macros

\newcommand{\tExt}{\MathAlgX{\widetilde{Ext}}}




\newcommand{\RO}{\MathAlgX{RO}}
\newcommand{\ROM}{\MathAlgX{ROM}}





%\newcommand{\com}{\Com}
\newcommand{\Comm}{{\textsf{Comm}}}
\newcommand{\Verify}{\MathAlgX{Verify}}
\newcommand{\Commit}{\MathAlgX{Commit}}
%\newcommand{\Open}{{\textsf{Open}}}

\newcommand{\tCom}{\MathAlgX{\widetilde{\Com}}}
\newcommand{\tGen}{\MathAlgX{\widetilde{\Gen}}}
\newcommand{\tVerify}{\MathAlgX{\widetilde{\Verify}}}
\newcommand{\tCommit}{\MathAlgX{\widetilde{\Commit}}}

\newcommand{\tEnc}{\MathAlgX{\widetilde{\Enc}}}
\newcommand{\tDec}{\MathAlgX{\widetilde{\Dec}}}




\newcommand{\Nins}{{\mathsf{No}}}
\newcommand{\Yins}{{\mathsf{Yes}}}
\newcommand{\Sins}{{\mathsf{Slack}}}



%%%%%%%





\newcommand{\fzk}{{\cF}_{\sf zk}}

%% ------------------ ZKP API
\newcommand{\ZK}{\textsf{ZK}}
\newcommand{\mom}{\mo^m}

\newcommand{\ipp}[1]{\iprod{#1}}


\newcommand{\DDH}{\MathAlgX{DDH}}
%%%ZK

\newcommand{\ZKP}{{\mathsf{ZK}}}
\newcommand{\fZKP}[1]{\MathAlgX{\cF^\ZKP_{#1}}}
\newcommand{\piZK}[1]{\MathAlgX{\Pi^\ZKP_{#1}}}
\newcommand{\vZK}[1]{\MathAlgX{\Vc^\ZKP_{#1}}}
\newcommand{\pZK}[1]{\MathAlgX{\Pc^\ZKP_{#1}}}


\newcommand{\ZKPOK}{{\mathsf{ZK\mhyphen POK}}}
\newcommand{\fZKPOK}[1]{\MathAlgX{\cF^\ZKPOK_{#1}}}
\newcommand{\piZKPOK}[1]{\MathAlgX{\Pi^\ZKPOK_{#1}}}
\newcommand{\vZKPOK}[1]{\MathAlgX{\Vc^\ZKPOK_{#1}}}
\newcommand{\pZKPOK}[1]{\MathAlgX{\Pc^\ZKPOK_{#1}}}







\newcommand{\pp}{{\mathsf {pp}}}


\newcommand{\ist}{{i^\ast}}
\newcommand{\bigPar}[1]{\big( #1\big)}

\newcommand{\Bad}{\mathsf{Bad}}

\newcommand{\skappa}{{\kappa_{\mathsf s}}}
\newcommand{\ckappa}{{\kappa_{\mathsf c}}}
\newcommand{\slack}{{\mathsf s}}
\newcommand{\sland}{\,\land\,}
\newcommand{\slor}{\,\lor\,}


\newcommand{\RSA}{{\MathAlgX{RSA}}}
%\newcommand{\IntCmt}{{\textsf{IntCom}}}



\newcommand{\crt}{\MathAlgX{crt}}
\newcommand{\jac}{\cJ}



\newcommand{\PaillierEnc}{\MathAlgX{PEnc}}
\newcommand{\PaillierDec}{\MathAlgX{PDec}}
\newcommand{\PaillierDom}[1]{\Z_{#1^2}^\ast}
\newcommand{\PaillierRndDom}[1]{\Z_{#1}^\ast}
\newcommand{\PaillierKeyGen}{\MathAlgX{PKeyGen}}
\newcommand{\PaillierKeyGenEx}{\MathAlgX{PPKeyGenEx}}



\newcommand{\IntCmtAlg}{\MathAlgX{IntCmt}}

\newcommand{\IntCmtParmGen}{\MathAlgX{Gen}}



\newcommand{\email}[1]{E-mail:\texttt{#1}.}


\newcommand{\ord}{\textsf{ord}}

 \newcommand{\cupdot}{\dot{\bigcup}}


%\newcommand{\divs}{\,|\,}
%\newcommand{\ndiv}{\,\cancel{\,|\,}\,}


\newcommand{\PedCom}{\mathsf{Ped}}

\newcommand{\hbeta}{\hat{\beta}}
\renewcommand{\R}{\mathbb{R}}

\newenvironment{indentedtext}{%
	\begin{list}{}{
			\setlength{\leftmargin}{0.1in}
		}
		\item
	}{%
	\end{list}
}

\mathchardef\mhyphen="2D


\newcommand{\Ber}{\mathrm{Ber}}




\newcommand{\Assert}[1]{{Assert(#1)}}




\mathchardef\mhyphen="2D
%\newcommand{\fZKPOK}[1]{\MathAlgX{\cF_{\sf zk\mhyphen pok}^{#1}}}



\newcommand{\pluseq}{\mathrel{+}=}
\newcommand{\minuseq}{\mathrel{-}=}
\newcommand{\unioneq}{\mathrel{\cup}=}

