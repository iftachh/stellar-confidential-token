
\newcommand{\init}{\mathsf{init}}
\newcommand{\mint}{\mathsf{mint}}
\newcommand{\transfer}{\mathsf{transfer}}
\newcommand{\audit}{\mathsf{audit}}
\newcommand{\rollover}{\mathsf{rollover}}
\newcommand{\withraw}{\mathsf{withraw}}
\newcommand{\history}{\mathsf{history}}
\newcommand{\balance}{\mathsf{avlBlance}}
\newcommand{\pendbalance}{\mathsf{pndBalance}}
\newcommand{\tcount}{\mathsf{tcount}}

\newcommand{\historylog}{\mathsf{log}}
\newcommand{\sConfTran}{\mathsf{ConfTrans}}
\newcommand{\fConfTran}{\MathAlgX{\cF_\sConfTran}}
\newcommand{\pConfTran}{\MathAlgX{\Pi_\sConfTran}}

\newcommand{\largest}{{\mathsf{lrg}}}
\newcommand{\posvst}{{\mathsf{pos}}}
\newcommand{\Chain}{{\MathAlgX{C}}}
\newcommand{\pcount}{p_{\mathsf{pcount}} }
\newcommand{\psize}{p_{\mathsf{size}} }


\newcommand{\rEnc}{{\cR_{\sf Enc}}}
\newcommand{\rEncDef}
{
	\set{((\pk,A),(a,r))\colon \Enc_\pk(a;r)= A}
}

\newcommand{\rKeyGen}{{\cR_{\sf KeyGen}}}

\newcommand{\rKeyGenDef}
{
	\set{(\pk,w))\colon \KeyGen(w)= (\cdot,\pk)}
}




\newcommand{\Eqs}{{\sf Eq}}
\newcommand{\rEq}{\cR_\Eqs}

\newcommand{\rEqDef}
{
	\set{((\pk_0,\pk_1,\oA_0,\oA_1),(a,r_0,r_1))\colon \forall i\in \zo \; \Enc_{\pk_i}(a;r_i)= \A_i}
}


\newcommand{\RPs}{{\sf Rp}}
\newcommand{\rRp}{\cR_\RPs}

\newcommand{\rRpDef}
{
	\set{((\pk,\oA,b),(a,r))\colon \Enc_\pk(a;r)= \oA \sland a\in (b)}
}


\newcommand{\LrgEqs}{{\sf LrgerEq}}
\newcommand{\rLrgEq}{\cR_\LrgEqs}

\newcommand{\rLrgEqDef}
{
	\set{((\pk,\oA_0,\oA_1),(a_0,r_0,a_1,r_1))\colon \forall i\in \zo \; \Enc_\pk(a_i;r_i)= \oA_i \sland a_1 -a_0 \in (q)} 
}


\section{The Confidential Transactions Protocol}\label{sec:MainProtocol}
In this section we define the confidential transactions scheme, and prove its security. The ideal functionality for the scheme is define in \cref{sec:MainProtocol:Ideal}, the protocol itself in \cref{sec:MainProtocol:Protocol}, and its security is proved in  \cref{prot:ConfidentialTransactions}.


\begin{remark}[$\sid$]\label{rem:sid}
Recall that all ideal functionalities operations below and the protocol execution get $\sid\in \zs$ as common input. The $\sid$ is stored in the log and passed as common input to the proofs. To keep the text simpler, we omit it from the following text.
\end{remark}


\subsection{The Ideal Functionality}\label{sec:MainProtocol:Ideal}
In this section we define the ideal functionality for the confidential transactions scheme. The functionality  captures the relevant parts of the actual scheme, which typically invoked using smart contracts over a block chain. Specifically, we model the  the chain as a single (honest) entity, the \textit{chain holder}, assume money flows into the system by a single (honest) party, the \textit{mint}, and assume fixed number of \textit{users}. We also assume an a initial starting phase,  \textit{init}.  

\begin{functionality}[\fConfTran: Confidential transactions]\label{func:ConfidentialTransactions}~
	
	
	
	\item[Parties:]	Mint \Mc, chain holder  \Cc and users  $\Uc_1, 
	\ldots,\Uc_n$.
	
	
	\item[Parameters:] $\pcount,\psize,q \in \N$.
	
	
	\begin{description}
		\item[Init.] Upon receiving $\init$ from  all parties: 
		\begin{enumerate}
			\item  For each $i\in [n]$:  $\balance_i\asn 0,\pendbalance_i \asn 0$, $\tcount_i \asn 0$,  $\historylog_i \asn \emptyset$.
			
			\item  $\historylog \asn \emptyset$.
		\end{enumerate}
		
			\item[Mint.]   Upon receiving $(\mint,d,x)$ from  \Cc and \Mc:
		\begin{enumerate}
			
			
			\item \Assert{$x\in (\psize) \sland \tcount_d\le  \pcount$}.
			
			\item $\tcount_d^{++}$.
			
			\item $\pendbalance_d \pluseq x$.
			
			\item Send $(\mint,d,x)$ to all parties.
		\end{enumerate}
		
		
		\item[Transfer.]    Upon receiving  $(\transfer,d)$ from  \Cc and  $\Uc_s$, with  $\Uc_s$ using private input $x$. 
		
	
		\begin{enumerate}
			\item \Assert{$ x\in (\psize) \sland \tcount\le  \pcount \sland \balance_s  \ge x $}.
			
				\item $\tcount^{++}$.
				
				
			\item $\balance_s \minuseq x$.
			
			\item   $\pendbalance_d \pluseq x$.
			
			
		%	\item  $\historylog_d \unioneq {(\transfer,s,x)}$
			
		%	\item  $\historylog \unioneq {(\transfer,s,d)}$ 
		
		\item Send $(\transfer,s,d)$ to all parties, and $x$ to $\Uc_d$.
			
		\end{enumerate}
		
		\item[Rollover.]     Upon receiving $\rollover$ from  party $\Uc_i$ and \Cc,  party \Cc
		
		\begin{enumerate}
			
			
			\item $\tcount \asn 0$.
			
			
			\item  $\balance_i  \pluseq \pendbalance_i$.
			
			\item  $\pendbalance_i  \asn 0$.
			
		%	\item  $\historylog \unioneq (\rollover,i)$
			
			
			\item  Send $ (\rollover,i)$ to all parties.
		\end{enumerate}

		
			\item[Withraw.]     Upon receiving $(\withraw,x)$ from  party $\Uc_i$ and \Cc,  party \Cc
		
		\begin{enumerate}
			
			
			 \item \Assert{$x\in (q) \sland \balance_i  \ge x$}.
			
			\item $\balance_i \minuseq x$.
			
			
			%\item  $\historylog \unioneq {(\withraw,i,x)}$
			
			\item Send $(\withraw,i,x)$ to all parties,
			
		\end{enumerate}
		
		
	%	\item[History.]     Upon receiving $\history$ from  party $\Uc_i$ and \Cc:   Send $(\historylog,\historylog_i)$ to  $\Uc_i$.	 
		
		\item[Audit.]      \Inote{TODO}
		
	\end{description}

\end{functionality}


\subsection{The Protocol}\label{sec:MainProtocol:Protocol}
Throughout, we fix a security parameter $\kappa$ and omit is from the notation. We also fix an homomorphic encryption scheme $(\KeyGen,\Enc,\Dec)$ over $\Z_q$ with randomness domain $\cD$, and denote the ciphertexts of the scheme using  overlined capital letters.   We require that $\Dec_{sk}(\oA)$ outputs   $(a;r)$ such that $\oA = \Enc(a;r)$.


We split  the protocol into several sub-protocols defined below, and  use the  following environment to define the common part the different sub-protocols share, \eg global parameter. 

\begin{protocol}[\pConfTran: Confidential transactions]~\label{prot:ConfidentialTransactions}


	
\item[Parties:] Mint \Mc, chain-holder \Cc and users  $\Uc_1, \ldots,\Uc_n$.

\item[Parameters:] $\pcount,\psize,q \in \N$.

\item[Subprotocols:]  See below.
\end{protocol}


\subsubsection{Init}


This sub-protocol is were the encryption key are sampled and shared, and the chain manager \Cc set the initial values of the chain. The protocol  uses ZKPOK proof for the relation:
\begin{description}
	\item[Key generation:]    $\rKeyGen =\rKeyGenDef$.
\end{description}


\begin{protocol}[\pConfTran.Init]~\label{prot:ConfidentialTransactions:Init}
	

	\item[Participating parties.] All parties.
			
			
	\item[Proofs:]   \piZKPOK{\rKeyGen}. 
	
%	\item[Common input:] $\sid \in \zs$.
	%\item[Algorithms:]    \KeyGen.
	
			
\item[Operation:] ~
			
			
			\begin{enumerate}
				
				\item   $\Uc_i$, for  all $i\in [n]$:
				\begin{enumerate}
					\item  $(\pk_i,\sk_i) \getsr \KeyGen(r_i)$ for $r_i \getsr \cD$.
					
					
					\item  $\pi_i \getsr \pZKPOK{\rKeyGen}(\pk_i,r_i)$.
					
					\item Send $(\pk_i,\pi_i)$ to \Cc.
					
					
				\end{enumerate}
				
				\item \Cc: ~
				
				\begin{enumerate}
				
				\item For all $i\in [n]$: 
				
				\begin{enumerate}
					\item 	 $\vZKPOK{\rKeyGen}(pk_i,\pi_i)$.\footnote{Here and after,  \Cc aborts and publish the prover identity if the   proof is not verified.}
					
					\item $\oP_i\getsr \Enc_{\pk_i}(0),\oB_i\asn \Enc_{\pk_i}(0),\tcount_i 
					\asn 0$.
					
			
				\end{enumerate}
			
		\item Broadcast $(\init,\sset{\pk_i,\oB_i,\oP_i}_{i\in [n]})$
		%	\item     Broadcast $\historylog \asn \emptyset$.
				\end{enumerate}								
			\end{enumerate}
\end{protocol}

\subsubsection{Mint}

\begin{protocol}[\pConfTran.Mint]~\label{prot:ConfidentialTransactions:Mint}
	\item[Parties:] \Mc and \Cc.
	
	
	
	\item[Common input:]  $d\in [n]$ and $x\in (\psize)$.
	
	\item[Operation:]  \Cc~

		\begin{enumerate}
			\item \Assert{$d\in [n] \sland \tcount_d  \le \pcount \sland x\in (\psize)$}
			
			\item  $\oP_d \pluseq \Enc_{\pk_d}(x)$.
				
	%	\item   $\historylog \unioneq (\mint,d,x,\tcount_d,\oP_d)$.
			
			\item Broadcast 	 	$(\mint,d,x,\oP_d)$.
		
	\end{enumerate}
\end{protocol}

\subsubsection{Transfer}

The protocol uses ZK and ZKPOK proofs for the   following  relations:

\begin{description}
	\item[In range.] $\rRp =\rRpDef$, \ie  encryption of values in $[\psize]$.
	
	\item[Equality.] $\rEq =\rEqDef$, \ie  encryptions of the same pair under different public keys.
	
	\item[Larger than.] $\rLrgEq =\rLrgEqDef$, \ie encryptions of the  pair of values $(a_0,a_1)$, under the same public key, with $a_1 \ge a_0$.
\end{description}




\begin{protocol}[\pConfTran.Transfer]~\label{prot:ConfidentialTransactions:Transfer}
	\item[Parties:] $\Uc_s$ and \Cc.
	
\item[Proofs:]   \piZK{\rRp},\pZK{\rEq},\pZK{\rLrgEq}


%\item[Algorithms:]    \Dec.

	
	\item[Common input:] $d\in [n]$.
	
	\item[$\Uc_s$'s private  input:] $x \in (\psize)$.
	
\item[Operation:] ~
	
	\begin{enumerate}
		\item  $\Uc_s$: 
		
		\begin{enumerate}
			\item $\oX_d \getsr \Enc_{\pk_d}(x;r_d)$ for $r_d \getsr \cD$.
			
		
			\item $\pi^\RPs \getsr \pZK{\rRp}((\pk_d,\oX_d,\psize),(x,r_d))$.
			
			
			\item $\oX_s \getsr \Enc_{\pk_s}(x;r)$ for $r_s \getsr \cD$.
			
			
			
			\item $\pi^\Eqs \getsr \pZK{\rEq}((\pk_s,\pk_d,\oX_s,\oX_s),(x,r_s,r_d))$.
			
			
			\item $(b,r^b) \asn \Dec_{sk_s}(\oB_s)$.
			
			
			\item $\pi^\LrgEqs \getsr \pZK{\rLrgEq}((\pk_s,\oX_s,\oB_s),(x,r_s,r_b))$.
			
			
			\item Send $(X_s,\oX_d,\pi^\RPs,\pi^\Eqs,\pi^\LrgEqs)$ to \Cc.
		\end{enumerate}
		
		
		\item  \Cc: 
		
		\begin{enumerate}
			
			\item  $\Assert{\tcount_d \le \pcount}$.
				
			\item   $\vZKPOK{\rRp}((\pk_d,\oX_d,\psize),\pi^\RPs)$,
			
			 $\vZKPOK{\Eqs}((\pk_s,\pk_d,\oX_s,\oX_s),\pi^\Eqs)$, $\vZKPOK{\rLrgEq}((\pk_s,\oX_s,\oB_s),\pi^\LrgEqs)$. 
			
	
		
		
		     \item   $\oB_s \minuseq \oX_s$. 
              
              \item  $\oP_d \pluseq X_d$. 
              
              \item $\tcount_d^{++}$.
              
		    % \item  $\historylog \unioneq (\transfer,s,d,\oB_s,\oP_d)$.
		     
		     \item Broadcast $(\transfer,s,d,\oB_s,\oP_d)$.
		\end{enumerate}
		
	\end{enumerate}
	
\end{protocol}



\subsubsection{Rollover}

\begin{protocol}[\pConfTran.Rollover]~\label{prot:ConfidentialTransactions:Rollover}
	\item[Parties.] $\Uc_i$ and \Cc.
	
\item[Operation:] \Cc:
	
	\begin{enumerate}
	
	\item $\oB_i  \pluseq \oP_i$.
	
	
	\item  $\oP_i \minuseq \oP_i$.
	
	\item  $\tcount_i \asn 0$.
	
	
	\item  Broadcast $(\rollover,i,\oB_i,\oP_i)$
	
	
\end{enumerate}
	
\end{protocol}

\remove{
\subsubsection{History}

\begin{protocol}[\pConfTran.History]~\label{prot:ConfidentialTransactions:History}
	\item[Parties.] $\Uc_i$ and \Cc.
		
	\item Operation: \Cc: Send  $\historylog$ to $\Uc_i$.
	
\end{protocol}
}

\subsubsection{Audit} 

\Inote{TODO}


\subsection{Security of \cref{prot:ConfidentialTransactions}}
\begin{theorem}[Security of \cref{prot:ConfidentialTransactions}]\label{thm:ConfidentialTransactions}
Assuming  $(\KeyGen,\Enc,\Dec)$ is CPA secure, then \cref{prot:ConfidentialTransactions} UC-realizes (with static security\Inote{?}) \cref{func:ConfidentialTransactions} against semi-honest chain holder and mint.
\end{theorem}

\begin{proof}
\Inote{TODO}
\end{proof}

