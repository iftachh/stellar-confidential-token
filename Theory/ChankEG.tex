\section{The ElGammal-Based  Additive Homomorphic Encryption }\label{sec:ChanksEG}
In this section we define the (efficient)  additive homomorphic encryption scheme based on ElGammal multiplicative  homomorphic encryption scheme.  \Inote{give citations}


The idea is to bootstrap the so-called   \textit{ElGammal in-the exponent}   additive homomorphic encryption  scheme,\footnote{It is called ElGamal ``in-the-exponent'' due to typical multiplicative group notation. Here use additive group notation, but keep the name for historical reason.}  which in turn is based on the  ElGammal multiplicative  homomorphic encryption scheme, that lacks efficient decryption algorithm, by splitting the plain text into small ``chunks''. That  is, we present message $m\in Z_t$ as

$\sum_{i \in (t/c)}   m_i \cdot 2^{i c}$, where $c$, the chunk size, is  some integer that divides $[t]$, and encrypt using    additive homomorphic EG each of the $m_i$. To decry $\oM= (M_0,\ldots,M_{t/c})$, one  
\begin{enumerate}
	\item Decrypt   each $M_i$ to get $m_i  \cdot G$.
	\item Use brute force to find $m$.\footnote{One can use standard   processing to speed-up this part from $c$ group operations to $\sqrt{c}$  operations, or even \Inote{cite} to   $\sqrt[3]{c}$.}
	
	\item Reconstruct $m$.
\end{enumerate}



\subsection{ElGammal In-the-Exponent Scheme}\label{sec:ChanksEG:EG}
\newcommand{\EgKG}{\MathAlgX{EgGen}}
\newcommand{\EgEnc}{\MathAlgX{EgEnc}}
\newcommand{\EgDec}{\MathAlgX{EgDec}}

Throughout we fix a cyclic  additive $q$-size group $\cG$ with generator $G$. The  ElGammal in-the-exponent scheme $(\EgKG,\EgEnc,\EgDec)$ is define as follows:
\begin{description}
	\item[Key generation:] $\EgKG()$ samples $e\getsr \Z_q$, and outputs $(e,e\cdot G)$.
	
	\item[Encryiption:] $\EgEnc_E(m)$  samples $e\getsr \Z_q$, and outputs  $\tM \asn (r\cdot G, r\cdot E + m \cdot G)$
	
	
		\item[Decription:] $\EgDec_e(\tM)$,   
		\begin{enumerate}
			\item 	Let $M \asn \tM_2 - e\cdot \tM_2$.
			
			\item Find (using brute force) $m$ so that $m\cdot G = M$.
			
			\item Output $m$.
		\end{enumerate}
	\end{description}
	
	
\subsection{The Scheme}\label{sec:ChanksEG:EGScheme}
In the following we fix $t,c\in \N$ with $t\le q$ and $\ell \asn t/c \in \N$. The encryption scheme $(\KeyGen,\Enc,\Dec)$ is defined as follows:

\begin{description}
	\item[Key generation:] $\KeyGen()$ 
	\begin{enumerate}
		\item For each $i\in (\ell)$:	sample $(e_i,E_i) \getsr \EgKG()$.
		
		\item Output $(\oe \asn (e_0,\ldots,e_{\ell-1}),\oE \asn (E_0,\ldots,R_{\ell-1})$
	\end{enumerate}

	
	\item[Encryiption:] $\Enc_{\oE}(m)$
	
	\begin{enumerate}
		\item Compute $m_0,\ldots,m_{\ell-1}$ do that $m =  \sum_{i \in (\ell)}   m_i \cdot 2^{i c}$.
		\item For each $i\in (\ell)$: let $\tM_i \getsr \EgEnc_{\oE_i}(m_i)$.
		
		\item Output $\oM \asn (\tM_0,\ldots, \tM_{\ell-1})$.
		
	\end{enumerate}
	 
	\item[Decription:] $\EgDec_{\oe}(\oM)$   
	\begin{enumerate}
		\item 	 For each $i\in (\ell)$: let $m_i \getsr \EgDec_{\oe_i}(\oM_i)$.
		
		\item Let  $m \asn \sum_{i \in (\ell)}   m_i \cdot 2^{i c}$.
		
		\item Output $m$.
	\end{enumerate}
\end{description}



\subsection{Proofs}\label{sec:ChanksEG:Proofs}

\paragraph{Knowledge of secret key.}   Proof \piZKPOK{\rKeyGen}. 

For each of the  the $\ell$ public keys, apply the (standard) ElGammal POK for the secret key relation.


\paragraph{Knowledge of plain text.}  
Protocol \piZKPOK{\rKeyGen}. 

\newcommand{\rEnc}{{\cR_{\sf enc}}}

\newcommand{\rEncDef}
{
	\set{((pk,A),(a,r))\colon \Enc_(pk)(a;r)= A}
}


Proof of knowledge for   the ciphertexts  relation  
$$\rEnc =\rEncDef.$$


On  plaintext $\oM$, for each $\oM_i$:
\begin{enumerate}
	\item Apply the  standard ElGammal POK for the ciphertexts relation. 
	
	\item Apply range proof, \ie \piZKPOK{\rRP}, to show that th encrypted plaintext is in $(2^c \cdot \psize)$
\end{enumerate}

 




\subsection{Adjusting  \cref{sec:MainProtocol:Protocol}}\label{sec:ChanksEG:Adjusting}
